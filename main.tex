\documentclass[11pt]{article}
\usepackage[letterpaper]{geometry}
\usepackage{times}
\geometry{top=1.0in, bottom=1.0in, left=1.0in, right=1.0in}
\usepackage{setspace}
%\doublespacing
\usepackage{wrapfig}
%\usepackage[document]{ragged2e}
%\setlength\RaggedRightParindent{4em}
\usepackage{fancyhdr}
\usepackage{fancyvrb} %Need for ventering verbatim using BVerbatim
\pagestyle{fancy}
\fancyhf{}
\rhead{Reichelt \thepage}
\renewcommand{\headrulewidth}{0pt}
\renewcommand{\footrulewidth}{0pt}
%To make sure we actually have header 0.5in away from top edge
%12pt is one-sixth of an inch. Subtract this from 0.5in to get headsep value
\setlength\headsep{0.333in}

\newcommand{\bibent}{\noindent \hangindent 40pt}
\newenvironment{workscited}{\newpage \begin{center} Works Cited \end{center}}{\newpage }
\usepackage{hyperref}
\usepackage{graphicx}
\usepackage{caption}
\usepackage[font=scriptsize,labelfont=bf, justification=centering]{caption}
\usepackage{xcolor}
\usepackage[group-separator={,}]{siunitx}
\newcommand{\TODO}[1]{\textcolor{red}{[#1]}}
\newcommand{\PASS}[1]{\verb| #1|}

\usepackage{titlesec}

\titleformat{\section}
  {\normalfont\fontsize{11}{11}\bfseries\scshape}{\thesection}{1em}{}

\titleformat{\subsection}
  {\normalfont\fontsize{9}{9}\bfseries\itshape}{\thesection}{1em}{}

\begin{document}

    \thispagestyle{empty}
    \begin{flushleft}
        Scott Reichelt\\
        Dr. Scott Herring\\
        UWP104E - Section 2\\
        \today\\
    \end{flushleft}
    \vspace{1em}
    \begin{center}
        Review of Originating Work on Whale Optimization Algorithm and Subsequent Applications
    \end{center}

\section*{Introduction}
\TODO{Talk about the more general Swarm Optimization Aglorithm Context to motivate WOA}
Since there introduction in \TODO{year, reference} with \TODO{author, algorithm} swarm algorithms have proven competetive with other meta-heuristic algorithms, and are attractive models because of their underlying simplicity.
Swarm algorithms are also attractive because they are relatively easy to implement, do not rely on knowledge of the gradient, can overcome problems posed my local optima, and can be adapted for a wide range of optimization problems (Mirjalili \& Lewis 2016).

\subsection*{WOA}
\TODO{Talk about WOA in specific terms, bridge gap between swarm algorithm and humpback whale stuff}

Swarm algorithms, since appearing in the 60's, have found a diverse set of applications in modern research.
Modern scale problems with vast amounts of available data can be approached using swarm algorithms, particularly when they are enhanced with the parallel computing techniques---i.e. partitioning the search space---that the algorithms easily accommodate (Cicirelli 2015).
Mirjalili has introduced a new approach called Whale Optimization Algorithm (WOA) to the field that is competitive both with canon swarm algorithms such as Particle Swarm Optimization (PSO) or Ant Colony Optimization (ACO), and other classes of meta-heuristic methods such has Genetic Algorithms (GA) (2016).
The WOA draws its inspiration from a hunting behavior recorded in rare populations of humpback whales called bubble-net feeding, which is a technique that allows groups of whales to efficiently corral and consume krill (Wiley et al. 2011).

\subsection*{Bubble-net Foraging: Cetological Background}
The hunting behavior which informs the WOA relies on a group of whales herding a school of fish and a single whale executing the bubble-net.
This whale will dive below the target prey and by exploiting the high maneuverability of humpback whales (Wiley et al. 2011) encircles a school of fish with a column of bubbles purposefully emanating from its blow hole.
The bubbles form an impassable barrier (Wiley et al. 2011) for the fish, and concentric circles corral target prey into a tight area either through an upward-spiral or double loop maneuver (Wiley et al. 2011).
The ring of bubbles forming the bubble-net can be seen from the surface of the water (figure 1), and as the column of bubbles corrals prey the team of humpback whales lunge from the columns center to feed as seen in figure 2.

\section*{WOA Applications and Evolution}
\subsection*{Data Mining}
\subsection*{Neural Net Feature Reduction}
\subsection*{Augmented WOA approaches}

\begin{workscited}

\bibent
Cicirelli, F. (2015). Strategies for Parallelizing Swarm Intelligence Algorithms. 23RD EUROMICRO INTERNATIONAL CONFERENCE ON PARALLEL, DISTRIBUTED, AND NETWORK-BASED PROCESSING (PDP 2015), 329-336.

\bibent
Sayed, G. (2017). Breast Cancer Diagnosis Approach Based on Meta-Heuristic Optimization Algorithm Inspired by the Bubble-Net Hunting Strategy of Whales. Genetic and Evolutionary Computing : Proceedings of the Tenth International Conference on Genetic and Evolutionary Computing, November 7-9, 2016 Fuzhou City, Fujian Province, China /, 536, 306-313.

\bibent
Mirjalili, S. (2016). The Whale Optimization Algorithm. Advances in Engineering Software, 95, 51-67.

\bibent
Ab Wahab, Mohd Nadhir, Samia Nefti-Meziani, and Adham Atyabi. “A Comprehensive Review of Swarm Optimization Algorithms.” Ed. Catalin Buiu. PLoS ONE 10.5 (2015): e0122827. PMC. Web. 19 Feb. 2018.

\bibent
Wiley, D. (2011). Underwater components of humpback whale bubble-net feeding behaviour. Behaviour., 148(5-6), 575-602.

\bibent
Rohani, M. (2016). THE WORKFLOW PLANNING OF CONSTRUCTION SITES USING WHALE OPTIMIZATION ALGORITHM (WOA). TURKISH ONLINE JOURNAL OF DESIGN ART AND COMMUNICATION, 6, 2938-2950.

\bibent %figure 1
Two humpback whales bubble net feeding. Image collected under MMPA research permit \#17355.
Credit: NOAA Fisheries/Allison Henry

\bibent %figure 2
By Evadb; Edit by jjron. (Own work) [Public domain], via Wikimedia Commons

\end{workscited}

\end{document}
\}
